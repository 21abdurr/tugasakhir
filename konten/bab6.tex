%-----------------------------------------------------------------------------------------------%
%
% Maret 2019
% Template Latex untuk Tugas Akhir Program Studi Sistem informasi ini
% dikembangkan oleh Inggih Permana (inggihjava@gmail.com)
%
% Template ini dikembangkan dari template yang dibuat oleh Andreas Febrian (Fasilkom UI 2003).
%
% Orang yang cerdas adalah orang yang paling banyak mengingat kematian.
%
%-----------------------------------------------------------------------------------------------%


%-----------------------------------------------------------------------------%
\chapter{\babEnam}
%-----------------------------------------------------------------------------%
\section{Kesimpulan}
Berdasarkan penelitian dan pembuatan sistem yang telah dilakukan, dapat disimpulkan sebagai berikut :

\begin{enumerate}
	\item Pembuatan Sistem Informasi Tatakelola Surat Berbasis Website Pada Dinas Kesehatan Kabupaten Pelalawan dapat digunakan untuk mempermudah tatakelola surat dan pengarsipan pada subbagian umum dan kepegawaian seperti surat masuk, surat keluar serta arsip surat.

	\item Hasil pengujian dengan metode \textit{Black Box} seluruh fungsi telah sesuai dengan sistem yang di usulkan dan diharapkan oleh pengguna sistem yaitu dengan persentase keberhasilan 100\%.

	\item Berdasarkan hasil pengujian metode \textit{User Acceptence Test} (UAT) didapatkan nilai dari pengujian sebesar 89\% dan dapat disimpulkan bahwa aplikasi berjalan dengan baik dan dapat membantu pengguna di Dinas Kesehatan Kabupaten Pelalawan.
	
\end{enumerate}
\section{Saran}
	Dari Kesimpulan yang telah diurakan diatas, maka ada beberapa saran yang
diperlukan, yaitu :
\begin{enumerate}
	
	\item Diharapkan \namasistem \ ini dapat di implementasikan di \namatempat
	\item Sistem yang akan diimplementasikan ini diharapkan dapat digunakan
	sebaik mungkin dalam melaksanakan kegiatan pelayanan kepada pengguna.
	\item Agar adanya penelitian lanjutan dimasa mendatang agar sistem ini dapat
	lebih disempurnakan
\end{enumerate}

