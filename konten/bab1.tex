%-----------------------------------------------------------------------------------------------%
%
% Maret 2019
% Template Latex untuk Tugas Akhir Program Studi Sistem informasi ini
% dikembangkan oleh Inggih Permana (inggihjava@gmail.com)
%
% Template ini dikembangkan dari template yang dibuat oleh Andreas Febrian (Fasilkom UI 2003).
%
% Orang yang cerdas adalah orang yang paling banyak mengingat kematian.
%
%-----------------------------------------------------------------------------------------------%

%-----------------------------------------------------------------------------%
\chapter{\babSatu}
%-----------------------------------------------------------------------------%

%-----------------------------------------------------------------------------%
\section{Latar Belakang}
%-----------------------------------------------------------------------------%
Surat adalah alat atau sarana komunikasi yang baik dalam bentuk tulisan maupun gambar yang digunakan oleh pihak-pihak terkait seperti perusahaan, organisasi, maupun pribadi kepada pihak lain untuk menyampaikan suatu informasi yang berfungsi sebagai bukti konkrit pada suatu hal atau kejadian tertentu \cite{triyono2013pembuatan}. Dalam suatu organisasi/perusahaan surat menurut prosedur pengurusannya dibagi menjadi dua yaitu surat masuk dan surat keluar. Surat masuk merupakan komunikasi tertulis berupa semua jenis surat yang diterima dari perusahaan atau instansi lain kepada pihak penerima. Surat masuk merupakan semua jenis surat yang diterima dari instansi lain maupun perorangan, baik yang diterima melalui pos maupun yang diterima melalui kurir dengan mempergunakan buku pengiriman/ekspedisi, sedangkan surat keluar adalah surat yang sudah lengkap (bertanggal, bernomor, berstempel, dan telah ditanda tangani oleh pejabat yang berwenang) yang dibuat oleh suatu instansi, kantor atau lembaga untuk ditujukan atau dikirim kepada instansi, kantor atau lembaga lain \cite{suherman2017sistem}.

Prosedur pengelolaan surat masuk meliputi; pengelompokan surat, membuka surat, pemerikasaan surat, pencatatan surat dan pendistribusian surat, sedangkan untuk surat keluar meliputi; pembuatan konsep, persetujuan konsep, pengertian surat, pemberian nomor, penyusunan surat, pengiriman surat. Prosedur pengolahan surat perlu diterapkan untuk masing-masing unit organisasi, karena merupakan sumber data atau informasi yang bermanfaat untuk kemajuan organisasi tersebut secara maksimal.

Kegiatan atau pekerjaan kantor yang berhubungan dengan penyimpanan dan pengelolaan warkat, surat surat dan dokumen - dokumen ini disebut kearsipan. Kearsipan memegang peranan penting bagi kelancaran suatu organisasi.

Sebagai salah satu kantor pemerintah yang tidak terlepas dengan kegiatan surat menyurat sebagai sarana komunikasi dengan pihak internal dan eksternal organisasi, penatausahaan surat dan arsip sangat dibutuhkan sebagai kegiatan pendukung bagi pelaksanaan tugas pokok Kantor Dinas Kesehatan Kabupaten Pelalawan (DINKES) yang merupakan instansi di daerah yang berhubungan langsung dengan satuan kerja dibidang advokasi kesehatan. Walaupun bukan merupakan pokok pelayanan organisasi, kegiatan ini menjadi sangat penting disebabkan dapat menjadi salah satu tolok ukur/indicator kinerja DINKES terhadap pemangku kepentingan.

Saat ini, di DINKES terdapat beberapa aplikasi yang dapat digunakan dalam penatausahaan surat dan arsip. Pada pelaksanaannya, penatausahaan surat belum memanfaatkan aplikasi penatausahaan surat dan masih dilakukan secara manual, disisi lain pemanfaatan aplikasi arsip belum digunakan secara menyeluruh di setiap unit kerja di DINKES yang berdampak pada penatausahaan arsip kurang efisien baik arsip fisik maupun arsip elektronik. Selain itu aplikasi yang ada tidak mengakomodasi alur proses yang melibatkan bagian-bagian di DINKES kebutuhan setiap bagian dalam pemantauan penyelesaian surat keluar. Penatausahaan dengan cara manual selama ini memiliki beberapa keterbatasan sebagai berikut :
\begin{enumerate}
\item Manajemen surat kurang efisien disebabkan waktu yang dibutuhkan dalam pencatatan secara manual dan distribusi fisik surat kepada masing-masing unit kerja
\item Terjadi duplikasi data dan fungsi, hal ini disebabkan masing-masing bagian melakukan penatausahaan arsip tersendiri baik arsip elektronik maupun arsip fisik,
\item Kesulitan dalam pencarian surat untuk keperluan referensi disebabkan arsip surat dan data elektronik surat keluar belum dikelola dengan baik.
\item Pengawasan kemajuan penerbitan surat keluar dan penyelesaian surat yang dapat dihubungkan dengan pengawasan kinerja pegawai tidak dapat dilakukan dengan baik.
\end{enumerate}

Pengembangan sistem informasi penatausahaan surat dan arsip untuk instansi DINKES memang telah banyak dilakukan. Tetapi sistem informasi yang dikembangkan tidak memperhatikan proses bisnis yang melibatkan berbagai bagian pada DINKES dan rata-rata bersifat standalone. Oleh karenanya dibutuhkan dibutuhkan pengembangan sistem informasi baru yaitu “\namasistem \ (\singkatansistem)” yang digunakan untuk menatausahakan surat yang mengakomodasi alur proses dan pengawasan kemajuan penerbitan surat dan penyelesaian surat dalam rangka pengawasan kinerja. Sitsar merupakan aplikasi berbasis web yang dikembangkan dengan Bahasa pemrograman PHP dengan pemilihan basis data MySQL. PHP dipilih karena kemudahannya, cepat dan bersifat multiplatform. Sedangkan MySQL merupakan basis Data yang digunakan pada aplikasi-aplikasi DINKES.
Dengan latar belakang di atas, menjadi dasar pertimbangan penulis untuk membuat laporan penelitian tugas akhir ini dengan mengangkat judul \textbf{“\judulkecil.”} 



%-----------------------------------------------------------------------------%
\section{Perumusan Masalah}
%-----------------------------------------------------------------------------%
Rumusan masalah penelitian ini adalah : \rumusanmasalah \ ?


%-----------------------------------------------------------------------------%
\section{Batasan Masalah}
%-----------------------------------------------------------------------------%
Adapun Batasan masalaah yang terdapat dalam penelitian kali ini sebagai berikut:
\begin{enumerate}
	\item Sistem yang dikembangkan adalah untuk mempermudah tatakelola surat dan arsip pada dinas kesehatan kabupaten pelalawan.
	\item Sistem informasi tatakelola surat dan arsip ini hanya membahas tentang surat masuk, surat keluar serta arsip surat.
	\item Sistem informasi tatakelola surat ini dibangun menggunakan Php sebagai Bahasa pemrograman dan Mysql sebagai database.
	\item Pengembangan system menggunakan metode waterfall
	\item Sistem ini menggunakan \textit{Unifield Modelling Language} (UML) sebagai
	toolsnya.
	\item Pengujian Sistem Menggunakan \textit{Black Box} Testing.
\end{enumerate}

%-----------------------------------------------------------------------------%
\section{Tujuan}
%-----------------------------------------------------------------------------%
Adapun tujuan dari penelitian tugas akhir ini adalah:
\begin{enumerate}
	\item Merancang sebuah sistem yang sudah terkomputerisasi untuk mendukung kebutuhan informasi mengenai pengolahan data tatakelola surat dan pengarsipan pada Subbagian Umum dan Kepegawaian pada Dinas Kesehatan Kabupaten Pelalawan.
	\item Menganalisa objek dan data pendukungnya untuk pembuatan sistem informasi arsip dokumen pada Subbagian Umum dan Kepegawaian pada Dinas Kesehatan Kabupaten Pelalawan.
	\item Untuk memudahkan pekerjaan dalam mencari atau penginputan surat dan arsip.
\end{enumerate}

%-----------------------------------------------------------------------------%
\section{Manfaat}
%-----------------------------------------------------------------------------%
Manfaat tugas akhir ini adalah:
\begin{enumerate}
	\item Dapat meminimalisir permasalahan yang terjadi pada Subbagian Umum dan Kepegawaian pada Dinas Kesehatan Kabupaten Pelalawan.
	\item Dapat membantu Subbagian Umum dan Kepegawaian dalam pengendalian tatakelola surat dan pengarsipan surat, baik itu dari segi waktu, tenaga dan juga biaya.
	\item Dapat memudahkan Dinas Kesehatan Kabupaten Pelalawan dalam memprioritaskan surat menurut kepentingan surat tersebut.
\end{enumerate}

%-----------------------------------------------------------------------------%
\section{Sistematika Penulisan}
%-----------------------------------------------------------------------------%
Sistematika penulisan laporan adalah sebagai berikut:

\textbf{BAB 1. \babSatu}

BAB 1 pada tugas akhir ini berisi tentang: (1) latar belakang masalah; (2) rumusan masalah; (3) batasan masalah; (4) tujuan; (5) manfaat; dan (6) sistematika penulisan.

\textbf{BAB 2. \babDua}

BAB 2 pada tugas akhir ini berisi tentang: tentang teori-teori yang berasal dari jurnal, buku serta studi kepustakaan yang digunakan sebagai landasan teori dalam pembuatan tugas akhir ini seperti tentang pengertian sistem informasi, manajemen data dan pengawasan.


\textbf{BAB 3. \babTiga}

BAB 3 pada tugas akhir ini berisi tentang: tentang metodologi atau urutan tata cara dan langkah-langkah penelitian dari tahap persiapan sampai dengan tahap mengembangkan sistem pengawasan.


\textbf{BAB 4. \babEmpat}

BAB 4 pada tugas akhir ini berisi tentang: penjelaskan tentang uraian permasalahan, analisis permasalahan dan perancangan sistem tahap dimana akan dibuat flowchart Alur Sistem Informasi menggunakan tool – tool UML (United Modelling Languange). Diagram-diagram yang digunakan dalam UML (United Modelling Languange) : usecase, sequence, class, activity.

\textbf{BAB 5. \babLima}

BAB 5 pada tugas akhir ini berisi tentang: penjelasan mengenai batasan implementasi, lingkungan implementasi dan hasil dari implementasi. Serta menjelaskan pengujian perangkat lunak dan hasil pengujian.

\textbf{BAB 6. \babEnam}

BAB 6 pada tugas akhir ini berisi tentang: tentang kesimpulan dan saran dari penelitian Tugas Akhir.