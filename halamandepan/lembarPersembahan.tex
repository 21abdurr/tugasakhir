%-----------------------------------------------------------------------------------------------%
%
% Maret 2019
% Template Latex untuk Tugas Akhir Program Studi Sistem informasi ini
% dikembangkan oleh Inggih Permana (inggihjava@gmail.com)
%
% Template ini dikembangkan dari template yang dibuat oleh Andreas Febrian (Fasilkom UI 2003).
%
% Orang yang cerdas adalah orang yang paling banyak mengingat kematian.
%
%-----------------------------------------------------------------------------------------------%

\chapter*{LEMBAR PERSEMBAHAN}
\begin{figure}
	\centering
	\includegraphics [height=2cm, width=8cm]{konten/gambar/quran.png}
\end{figure}

“Bacalah dengan menyebut nama Tuhanmu
Dia telah menciptakan manusia dari segumpal darah
Bacalah, dan Tuhanmulah yang maha mulia
Yang mengajar manusia dengan pena, Dia mengajarkan manusia apa yang tidak diketahuinya..”

\begin{center}
	\centering (QS. Al-‘Alaq1-5)
\end{center}

\begin{center}
	Alhamdulillah, Alhamdulillah, Alhamdulillahirobbil’alamin. Segala puji bagi Allah SWT, Tuhan yang Maha Agung dan Maha Tinggi.Sujud syukur kupersembahkan kepada-Mu, dengan Rahmat dan Rahim-Mu telah kau jadikan aku manusia yang senantiasa berpikir, berilmu, beriman dan bersabar dalam menjalani kehidupan ini.Semoga keberhasilan ini menjadi satu langkah awal bagiku untuk meraih cita-cita besarku.
\end{center}
\begin{center}
	Dengan lantunan Al-fatihah beriring shalawat serta menadahkan tangan didalam doa, terimakasih kepersembahkan untuk-Mu. Kupersembahkan karya kecil ini sebagai tanda bakti, hormat, dan rasa terimakasih yang tiada terhingga kepada Ibu dan Ayah yang telah memberikan kasih sayang, segala dukungan, dan cinta kasih yang tiada terhingga yang tiada mungkin dapat kubalas hanya dengan selembar kertas yang bertuliskan kata cinta dan persembahan. Semoga ini menjadi langkah awal untuk membuat Ibu dan Ayah bahagia.
\end{center}
\begin{center}
	Ayahanda Agus Zaini dan Ibunda Arnihar  tercinta, terimakasih....
\end{center}
\begin{center}
	Yaa Allah 
\end{center}
\begin{center}
	berikanlah balasan setimpal syurga firdaus untuk mereka dan jauhkanlah mereka dari panasnya sengat hawa apineraka-Mu…
	Amiiiin yaa Rabbal’alamin...Teruntuk Ayah anda dan Ibunda Tercinta..
\end{center}

\begin{center}
	Abdur Rahman
\end{center}